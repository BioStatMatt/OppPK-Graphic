\documentclass{article}
\usepackage{amsmath}
\begin{document}
\section{Two-compartment model}
The two-compartment pharmacokinetic model is expressed as a system of two ordinary differential equations as follows, where $m_1$ and $m_2$ are the masses of drug in the central and peripheral compartments, respectively.

\begin{align}
\frac{dm_1}{dt} &= -k_{10}m_1 - k_{12}m_1 + k_{21}m_2 + k_R \nonumber \\
\frac{dm_2}{dt} &= \phantom{-k_{10}m_1} + k_{12}m_1 - k_{21}m_2 \nonumber
\end{align}

The concentration of drug in the central compartment is given by $c_1 = m_1/v_1$, where $v_1$ is the volume of the central compartment. The parameters $k_{10}$, $k_{12}$, $k_{21}$, and $k_R$ are described in Table \ref{tab:pkpars}

\begin{table}
\begin{tabular}{lll} \hline
Parameter & Units & Description \\ \hline
$k_{10}$ & h$^{-1}$ & Elimination rate from central compartment\\
$k_{12}$ & h$^{-1}$ & Distribution rate from central to peripheral compartment\\
$k_{21}$ & h$^{-1}$ & Distribution rate from peripheral to central compartment\\
$k_R$  & g$\cdot$h$^{-1}$ & Infusion rate into central compartment\\
$v_1$  & l & Volume of central compartment\\
\hline
\end{tabular}
\caption{Two-compartment model parameters. \label{tab:pkpars}}
\end{table}

\section{Bayes prediction model}
\end{document}
