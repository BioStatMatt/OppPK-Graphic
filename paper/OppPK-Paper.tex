\documentclass{article}
\usepackage{amsmath}
\begin{document}

% Outline:
% 1. Introduction
% 2. Methods
%  - PK model (two-compartment)
%  - PD summaries (T>MIC)
%  - Statistical model (Bayesian NLR)
% Quality of predictions (computing efficiency, 'coverage' of credible intervals)
% Extensions
% Conclusions

\section{Introduction}
\begin{itemize}
\item Goal is to facilitate real-time herapeutic drug monitoring (TDM); translational research.
\item PK predictions often not reported with uncertainty; cite literature.
\item Need to quantify uncertianty in real time, but it's difficult.
\item incorporate objective prior information
\item We created a web app and suite of methods that accomplish this.
\end{itemize}

[need references]
Adverse consequences can arise from both antibiotic underdosing and overdosing  of hospitalized patients. In the former case, patients receive an insufficient amount of antibiotic to effectively combat the infection. In the latter, an overabundance of antibiotic often increases neurotoxicity risks and length of stay. In another study [cite], we showed roughly 20\% of ICU patients being treated for sepsis were underdosed, while about 15\% were overdosed.  Variation in patient complications, such as multiple failing organs, can influence the dosing [and metabolism?] of an antibiotic. Simple dosing strategies (e.g. based on creatinine, alone) are unable to accurately capture the ideal amount of an antibiotic required for safe and effective treatment since individual pharmacokinetic heterogeneity varies greatly between patients.\\

The proposed statistical methodology and tools are designed to estimate how an individual is responding to treatment as measured by the concentration of antibiotic in the blood over time, predict how that individual will continue to respond, and determine how treatment can be adapted to reduce the risk of adverse outcomes.  The goal is to design an effective procedure of therapeutic drug monitoring (TDM) to perform this assessment accurately and in real time. Dosing for a patient needs to be enough that the concentration in the blood remains above the necessary threshold for suppression of the offending microorganism (usually a multiple of the minimum inhibitory concentration), yet not so high as to increase the risk of adverse side effects.\\
[do we need to mention or refrain from mentioning that the plan for a crossover study is in the works]\\

Furthermore, predictions of pharmacokinetic performance often fail to provide estimates of uncertainty [cite literature]. We implement Bayesian methods using pharmacokinetic data from a prior study [cite] to provide posterior estimates of individual  target attainment, measured as the fraction of the dosing period spent above the necessary blood concentration threshold. The fact that patients tend to vary widely in their response to treatment indicates that measures of uncertainty are critical to understanding response to treatment at both the population and patient-specific level. In order for the implementation of such procedures to be truly beneficial to individuals, it is necessary to have a method of assessing and predicting patient response in real time. This introduces the issue of selecting a statistical approximation technique that is both accurate in its quantification of uncertainty while also being computationally feasible to implement via a web application. Our statistical methodology is paired with a web application that implements these techniques and makes them accessible to, for example, a physician in real time. Assessments of the performance and computational efficiency are presented for various approximation methods.

\section{Methods}
\subsection{Two-compartment model}
The two-compartment pharmacokinetic model is expressed as a system of two ordinary differential equations as follows, where $m_1$ and $m_2$ are the masses of drug in the central and peripheral compartments, respectively.

\begin{align}
\frac{dm_1}{dt} &= -k_{10}m_1 - k_{12}m_1 + k_{21}m_2 + k_R \nonumber \\
\frac{dm_2}{dt} &= \phantom{-k_{10}m_1} + k_{12}m_1 - k_{21}m_2 \nonumber
\end{align}

The concentration of drug in the central compartment is given by $c_1 = m_1/v_1$, where $v_1$ is the volume of the central compartment. The parameters $k_{10}$, $k_{12}$, $k_{21}$, and $k_R$ are described in Table \ref{tab:pkpars}

\begin{table}
\begin{tabular}{lll} \hline
Parameter & Units & Description \\ \hline
$k_{10}$ & h$^{-1}$ & Elimination rate from central compartment\\
$k_{12}$ & h$^{-1}$ & Distribution rate from central to peripheral compartment\\
$k_{21}$ & h$^{-1}$ & Distribution rate from peripheral to central compartment\\
$k_R$  & g$\cdot$h$^{-1}$ & Infusion rate into central compartment\\
$v_1$  & L & Volume of central compartment\\
\hline
\end{tabular}
\caption{Two-compartment model parameter units (SI) and descriptions. \label{tab:pkpars}}
\end{table}

\subsection{Bayes prediction model}
Concentration measurements are modeled using a nonlinear regression method with either additive error as follows:
\begin{displaymath}
c_{ij} = \eta_i(t_{ij}, \theta_i) + \epsilon_{ij}
\end{displaymath}
\noindent In this expression, $c_{ij}$ is the measured concentration for subject $i = 1 \ldots n$ at time $t_{ij}$ for $j = 1 \ldots m_i$, $\eta_i(t_{ij}, \theta_i)$ is the two-compartment model solution for subject $i$ at time $t_{ij}$ given parameters $\theta_i = [k_{10i}, k_{12i}, k_{21i}, v_{1i}]$, and $\epsilon_{ij}$ represents i.i.d. random error with mean zero. By specifying the random error density function, say $f(\epsilon_{ij}, \sigma)$, the subject-specific likelihood function is 
\begin{displaymath}
L_i(\theta_i, \sigma) = \prod_{j=1}^{m_i} f(c_{ij} - \eta_i(t_{ij}, \theta_i), \sigma).
\end{displaymath}
\noindent Thus, given a prior distribution $\pi_0(\theta_i)$, the subject-specific posterior is proportional to $\pi_i(\theta_i, \sigma) \propto \pi_0(\theta_i,\sigma)L_i(\theta_i,\sigma)$.

In the current context, $f$ is taken as the normal density function with mean zero and standard deviation $\sigma$, and the prior distribution is generated to satisfy the following:
\begin{align}
[\log \theta_i] &\sim N_4(\mu_0, \Sigma_0) \\
[\log \sigma] &\sim N_1(m_0, s_0)
\end{align}
\noindent where $N_4(\mu_0, \Sigma_0)$ represents the 4-variate normal distribution with mean $\mu_0$ and covariance matrix $\Sigma_0$, and $N_1(m_0, s_0) represents the univariate normal distribution with mean $m_0$ and variance $s_0$. The prior distribution for the PK parameters represents our prior knowledge about the PK heterogeneity for a particular drug in a target population. Thus, the values of the prior hyperparameters should be carefully selected for the task at hand. In the current context, and by default, the hyperparameters are specified to correspond with estimates that arose from a study of piperacillin pharmacokinetics in a hospitalized, critically ill population. The particular values are listed in an appendix. 

Due to the nonlinearity of the two-compartment model, the posterior distribution does not take a familiar form, and posterior summaries must be approximated. In particular, we sought to compute 95% credible bands for subject-specific concentration-time curves, and for a specific summary of the curve: the length of time in which the concentration exceeds a specified value. Monte Carlo techniques are often used to approximate these quantities. However, because these posterior summaries are presented in a web application in real time, we sought alternatives that were less computationally intensive and deterministic. We consider several methods as described in the following sections.

\subsubsection{Laplace approximation}
We first considered a method that makes use of two approximations. The first is a Laplace approximation to the subject-specific posterior density, and the second is a first order Taylor approximation of the target summary (i.e., the `delta method'). The Laplace approximation is given as follows
\begin{displaymath}
\pi_i(\log \theta_i, \log \sigma) \approx N([\log \hat{\theta_i},\log \hat{\sigma}], [-H]^{-1})
\end{displaymath} 
\noindent where $[\log \hat{\theta_i},\log \hat{\sigma}]$ is the posterior mode and $H_{\hat{\theta_i}}$ is the posterior Hessian with respect to $[\log \theta_i, \log \sigma]$ evaluated at the posterior mode. The second approximation makes use of the delta method, such that a posterior functional $h(\log \theta_i)$ has an approximate normal distribution. In the present context, $h(\log \theta_i)$ represents the logit of the fraction of the dosing period in which the concentration of drug exceeds a spedified value. The first-order Taylor expansion of $h(\log \theta_i)$ is 
\begin{displaymath}
h(\log \theta_i) \approx  h(\log \hat{\theta_i}) + G^T(\log \theta_i - \log \hat{\theta_i}),
\end{displaymath}
\noindent where $G$ is the gradient of $h(\log \theta_i)$ with respect to $[\log \theta_i, \log \sigma]$ evaluated at the posterior mode. Thus, given that $[\log \theta_i, \log \sigma]$ has an approximate normal distribution, the same is true for $h(\log \theta_i)$:
\begin{displaymath}
h(\log \theta_i) \sim N(h(\log \hat{\theta_i}), G^T[-H]^{-1}G). 
\end{displaymath}
\noindent An approximate $(1-\alpha)\cdot 100$\% credible interval for $h(\log \theta_i)$ is thus given by the $\alpha/2$ and $(1-\alpha)/2$ quantiles of the approximate posterior distribution for $h(\log \theta)$. This method is computationally elegant, since the posterior Hessian and posterior mode can be computed simultaneously by most optimization software routines (e.g., the R function `optim'). Indeed, this method usually has the smallest computational burden (i.e., number of likelihood evaluations) among the three methods considered here.

\subsubsection{Variational approximation}
\subsubsection{Quadrature approximation}
\subsection{Evaluation of approximation methods}


The posterior coverage of the MIC statistic 95\% confidence interval was estimated as follows:
 
50 samples were drawn from the prior distribution of PK parameters. For each set of prior parameters, 6 observations were simulated using the concentration predicted by the first 4 PK parameters and the standard error term corresponding to the 5th prior parameter. All observations were simulated during the final (5th) dosing period at 32, 32.5, 33, 34, 36, and 38 hours. 

For each of the 50 sets of prior parameters and simulated data:
 
\begin{itemize}
	 \item A Laplace approximation to the posterior density was calculated and the delta method used to estimate the standard errors of the MIC statistic. A 95\% asymptotic normal confidence interval was used with the delta method standard errors to estimate upper and lower bounds.
	 \item 3,000 samples from the posterior distribution of PK parameters were taken using a Metropolis MCMC algorithm with 2,000 warm-up iterations. The proportion of the 3,000 posterior samples falling within the upper and lower bounds was used to estimate the proportion of the posterior distribution of the MIC statistic falling within the confidence interval. 
\end{itemize}
		
On 50 samples we calculated a mean posterior coverage of 89.87\% with an IQR 88.10\% to 92.39\%. We calculate a Monte-Carlo standard error of approximately $\sqrt{(0.9)(0.1)/3000} = 0.005$. Confidence intervals for the MIC statstic were also considered on the logit and probit scales in order to restrict the bounds to the interval [0,1]; however, as these transformations resulted in slightly lower and more variable posterior coverage, the original scale was used. 
		

\begin{itemize}
\item used MCMC as gold-standard
\item quantified posterior probability contained within approximate intervals for $\logit^{-1}(h(\log \theta_i))$
\item quantified number of posterior evaluations
\item quantified computing time
\item use ``plasmode'' simulation to assess quality of method across variety of individuals (PK heterogeneity in this pop); recent example of plasmode use in \cite{Franklin2014}; pioneered by \cite{Vaughan2009}
\end{itemize}
\section{Appendix: Prior hyperparameters for piperacillin among critically ill}
\begin{center}
\begin{table}
\begin{tabular}{lll} \hline
Parameter & Value \\ \hline
$\mu_0$ & \\
$\Sigma_0$ & \\
$\alpha_0$ & \\
$\beta_0$  & \\
\hline
\end{tabular}
\caption{Prior hyperparameters for piperacilling in a population of hospitalized, critically ill patients.}\label{tab:hyp}
\end{table}
\end{center}

\end{document}
